%% Generated by Sphinx.
\def\sphinxdocclass{report}
\documentclass[letterpaper,10pt,spanish]{sphinxmanual}
\ifdefined\pdfpxdimen
   \let\sphinxpxdimen\pdfpxdimen\else\newdimen\sphinxpxdimen
\fi \sphinxpxdimen=.75bp\relax

\PassOptionsToPackage{warn}{textcomp}
\usepackage[utf8]{inputenc}
\ifdefined\DeclareUnicodeCharacter
% support both utf8 and utf8x syntaxes
  \ifdefined\DeclareUnicodeCharacterAsOptional
    \def\sphinxDUC#1{\DeclareUnicodeCharacter{"#1}}
  \else
    \let\sphinxDUC\DeclareUnicodeCharacter
  \fi
  \sphinxDUC{00A0}{\nobreakspace}
  \sphinxDUC{2500}{\sphinxunichar{2500}}
  \sphinxDUC{2502}{\sphinxunichar{2502}}
  \sphinxDUC{2514}{\sphinxunichar{2514}}
  \sphinxDUC{251C}{\sphinxunichar{251C}}
  \sphinxDUC{2572}{\textbackslash}
\fi
\usepackage{cmap}
\usepackage[T1]{fontenc}
\usepackage{amsmath,amssymb,amstext}
\usepackage{babel}



\usepackage{times}
\expandafter\ifx\csname T@LGR\endcsname\relax
\else
% LGR was declared as font encoding
  \substitutefont{LGR}{\rmdefault}{cmr}
  \substitutefont{LGR}{\sfdefault}{cmss}
  \substitutefont{LGR}{\ttdefault}{cmtt}
\fi
\expandafter\ifx\csname T@X2\endcsname\relax
  \expandafter\ifx\csname T@T2A\endcsname\relax
  \else
  % T2A was declared as font encoding
    \substitutefont{T2A}{\rmdefault}{cmr}
    \substitutefont{T2A}{\sfdefault}{cmss}
    \substitutefont{T2A}{\ttdefault}{cmtt}
  \fi
\else
% X2 was declared as font encoding
  \substitutefont{X2}{\rmdefault}{cmr}
  \substitutefont{X2}{\sfdefault}{cmss}
  \substitutefont{X2}{\ttdefault}{cmtt}
\fi


\usepackage[Sonny]{fncychap}
\ChNameVar{\Large\normalfont\sffamily}
\ChTitleVar{\Large\normalfont\sffamily}
\usepackage{sphinx}

\fvset{fontsize=\small}
\usepackage{geometry}


% Include hyperref last.
\usepackage{hyperref}
% Fix anchor placement for figures with captions.
\usepackage{hypcap}% it must be loaded after hyperref.
% Set up styles of URL: it should be placed after hyperref.
\urlstyle{same}

\addto\captionsspanish{\renewcommand{\contentsname}{Indice:}}

\usepackage{sphinxmessages}
\setcounter{tocdepth}{1}



\title{DocName}
\date{29 de abril de 2021}
\release{Release 1.0}
\author{YourName}
\newcommand{\sphinxlogo}{\vbox{}}
\renewcommand{\releasename}{Versión}
\makeindex
\begin{document}

\ifdefined\shorthandoff
  \ifnum\catcode`\=\string=\active\shorthandoff{=}\fi
  \ifnum\catcode`\"=\active\shorthandoff{"}\fi
\fi

\pagestyle{empty}
\sphinxmaketitle
\pagestyle{plain}
\sphinxtableofcontents
\pagestyle{normal}
\phantomsection\label{\detokenize{index::doc}}


\noindent{\hspace*{\fill}\sphinxincludegraphics[width=500\sphinxpxdimen,height=500\sphinxpxdimen]{{bottlelogo}.png}\hspace*{\fill}}

\sphinxAtStartPar
En esta práctica vamos a trabajar con el framework Bottle visto en el Seminario 3.
La idea es crear un servicio web de directorio similar al \sphinxhref{https://directorio.uca.es/cau/directorio.do}{Directorio de la Universidad de Cádiz}.

\sphinxAtStartPar
El servicio deberá contar con un listado (evidentemente ficticio) del personal de la UCA, y cada miembro
contará con los siguientes atributos:
\begin{itemize}
\item {} 
\sphinxAtStartPar
DNI.

\item {} 
\sphinxAtStartPar
Nombre Completo.

\item {} 
\sphinxAtStartPar
Correo Electronico.

\item {} 
\sphinxAtStartPar
Departamento.

\item {} 
\sphinxAtStartPar
Categoria.

\item {} 
\sphinxAtStartPar
Lista de Asignaturas.

\end{itemize}

\sphinxAtStartPar
El servicio web deberá contar, al menos, con los siguientes \sphinxstylestrong{endpoints}:
\begin{itemize}
\item {} 
\sphinxAtStartPar
Dar de alta a un nuevo miembro.

\item {} 
\sphinxAtStartPar
Modificar los datos de un miembro.

\item {} 
\sphinxAtStartPar
Consultar la lista de todos los miembros de la Universidad.

\item {} 
\sphinxAtStartPar
Hacer una Búsqueda por DNI.

\item {} 
\sphinxAtStartPar
Obtener una lista de miembros según categoria.

\end{itemize}
\begin{description}
\item[{\sphinxstylestrong{Mejoras}}] \leavevmode\begin{itemize}
\item {} 
\sphinxAtStartPar
Crear un endpoint para eliminar un miembro mediante una operación HTTP DELETE

\item {} 
\sphinxAtStartPar
Utilizar alguna forma de persistencia de datos, almacenando la información de los miembros en unfichero o en una base de datos.

\item {} 
\sphinxAtStartPar
Habilitar una búsqueda parcial por nombre.

\item {} 
\sphinxAtStartPar
Habilitar una búsqueda paramétrica en la que se pueda elegir por qué criterio buscar.

\item {} 
\sphinxAtStartPar
Implementar una búsqueda inversa por asignaturas, listando los miembros PDI que pertenezcan a una asignatura.

\item {} 
\sphinxAtStartPar
Verificar que los datos introducidos al añadir un nuevo miembro son correctos, como por ejemplo la letra del DNI.

\item {} 
\sphinxAtStartPar
Otros endpoints adicionales que el estudiante considere oportunos.

\end{itemize}

\end{description}


\chapter{Instalación de Herramientas}
\label{\detokenize{documentos/instalacion:instalacion-de-herramientas}}\label{\detokenize{documentos/instalacion::doc}}

\section{GitHub}
\label{\detokenize{documentos/instalacion:github}}
\sphinxAtStartPar
Esta aplicación de búsqueda, modificación, registro y eliminación de usuarios se encuentra alojada en GitHub, para
entrar, acceder a través de este enlace: \sphinxhref{https://github.com/Mwazoski/Bottle-SD}{BottlePy}.

\sphinxAtStartPar
Una vez accedemos al repositorio, encontraremos un árbol de directorios
\begin{itemize}
\item {} 
\sphinxAtStartPar
docs

\item {} 
\sphinxAtStartPar
source

\end{itemize}
\begin{description}
\item[{Docs}] \leavevmode
\sphinxAtStartPar
Tiene todo lo relacionado con la documentación del software, tanto la generación automatica de la documentación interna
a través de la herramienta \sphinxhref{https://www.sphinx-doc.org/en/master/}{Sphynx}, que permite auto\sphinxhyphen{}generar código de Python siguiendo simplemente unos patrones de comentado
de código, como la generación del HTML y Latex del resto de tipos de documentación.

\sphinxAtStartPar
Dejando la documentación expuesta al público, se facilita para el próximo desarrollador de un proyecto abierto como este, la posibilidad
de seguir contribuyendo, no solo con la parte del código, si también documentando todas las nuevas características que se va añadiendo a
esta APIRest.

\sphinxAtStartPar
Todo esto tiene una intención meramente expositiva, puesto que este proyecto dudo que se siga desarrollando.

\item[{Source}] \leavevmode
\sphinxAtStartPar
En el directorio source encontramos:
\begin{itemize}
\item {} 
\sphinxAtStartPar
Los templates HTML

\item {} 
\sphinxAtStartPar
Base de Datos Sqlite3

\item {} 
\sphinxAtStartPar
Archivo de configuración \sphinxstylestrong{«requirements.txt»}

\item {} 
\sphinxAtStartPar
Scripts \sphinxstylestrong{Run.py} y \sphinxstylestrong{Runsql.py}

\end{itemize}

\end{description}


\section{Instalación}
\label{\detokenize{documentos/instalacion:instalacion}}
\sphinxAtStartPar
Lo primero que realizaremos para manejar esta app será un \sphinxtitleref{git clone}

\begin{sphinxVerbatim}[commandchars=\\\{\}]
\PYGZdl{} git clone https://github.com/Mwazoski/Bottle\PYGZhy{}SD.git
\end{sphinxVerbatim}

\sphinxAtStartPar
Una vez hallamos descargado el proyecto, entraremos dentro del directorio source, donde crearemos nuestro
entorno virtual para poder, más tarde, realizar la instalación de todos los módulos necesarios para manejar
tanto la documentación, como el código.

\sphinxAtStartPar
Creación de un entorno Virtual {\hyperref[\detokenize{documentos/instalacion:venv}]{\sphinxcrossref{Venv}}}


\bigskip\hrule\bigskip


\sphinxAtStartPar
Cuando hayamos creado el espacio virtual y estemos dentro de él, tendremos que instalar mediante la herramienta \sphinxstylestrong{pip}, todas las
dependencias que esta APIRest requiere. Simplemento ejecutamos el comando

\begin{sphinxVerbatim}[commandchars=\\\{\}]
\PYGZdl{} pip install \PYGZhy{}r requirements.txt
\end{sphinxVerbatim}

\sphinxAtStartPar
Este cogerá del archivo \sphinxstyleemphasis{requirements.txt} todas las dependencias que se requieren, includas la especificación de sus versiones.

\sphinxAtStartPar
Una vez hemos instalado todas las dependencias, estaremos listos para volver a la carpeta \sphinxstylestrong{/source} y poder lanzar el programa,
esto lo realizaremos mediante la invocación del intérprete python.

\begin{sphinxVerbatim}[commandchars=\\\{\}]
\PYGZdl{} python run.py
\end{sphinxVerbatim}


\subsection{Venv}
\label{\detokenize{documentos/instalacion:venv}}
\sphinxAtStartPar
\sphinxstylestrong{Introducción}

\sphinxAtStartPar
Un entorno virtual de Python es un ambiente creado con el objetivo de aislar recursos como librerías y entorno de ejecución,
del sistema principal o de otros entornos virtuales. Lo anterior significa que en el mismo sistema, maquina o computadora,
es posible tener instaladas multiples versiones de una misma librería sin crear ningún tipo de conflicto.

\sphinxAtStartPar
Para poder utilizar este simple pero poderoso concepto es necesario instalar una utilidad que permita gestionar la creación
y utilización de dichos entornos virtuales llamada venv.

\sphinxAtStartPar
El módulo venv proporciona soporte para crear «entornos virtuales» ligeros con sus propios directorios de ubicación,
aislados opcionalmente de los directorios de ubicación del sistema. Cada entorno virtual tiene su propio binario Python
(que coincide con la versión del binario que se utilizó para crear este entorno) y puede tener su propio conjunto independiente
de paquetes Python instalados en sus directorios de ubicación.

\sphinxAtStartPar
\sphinxstylestrong{Como usar Venv}

\begin{sphinxVerbatim}[commandchars=\\\{\}]
\PYG{c+c1}{\PYGZsh{} python3 venv carpeta\PYGZus{}destino}

\PYGZdl{} python3 venv mientorno
\end{sphinxVerbatim}

\sphinxAtStartPar
Una vez se ha creado el entorno virtual, debemos entrar en dicho entorno.

\begin{sphinxVerbatim}[commandchars=\\\{\}]
\PYGZdl{} \PYG{n+nb}{source} mientorno/bin/activate
\end{sphinxVerbatim}

\sphinxAtStartPar
Una vez ejecutado este comando, estaremos dentro del entorno virtual, sabremos esto ya que nuestra consola
ahora aparecerá de la siguiente manera.

\begin{sphinxVerbatim}[commandchars=\\\{\}]
\PYG{o}{(}mientorno\PYG{o}{)} user@user:\PYGZti{}/
\end{sphinxVerbatim}


\section{Add\sphinxhyphen{}Ons}
\label{\detokenize{documentos/instalacion:add-ons}}
\sphinxAtStartPar
Esta APIRest desbloquea mucha de sus funciones, gracias al usuario administrador, el cual es capaz de realizar acciones
adicionales, por defecto, el usuario adminisitrador:
\begin{description}
\item[{Admin}] \leavevmode\begin{itemize}
\item {} 
\sphinxAtStartPar
\sphinxstylestrong{Usuario:} admin

\item {} 
\sphinxAtStartPar
\sphinxstylestrong{Correo Electrónico:} \sphinxhref{mailto:admin@uca.es}{admin@uca.es}

\item {} 
\sphinxAtStartPar
\sphinxstylestrong{Contraseña:} admin

\end{itemize}

\end{description}

\sphinxAtStartPar
Logeandonos con estas credenciales podremos obtener acceso a diferentes funcionalidades como
\begin{itemize}
\item {} 
\sphinxAtStartPar
\sphinxstyleemphasis{Eliminar Usuarios}

\item {} 
\sphinxAtStartPar
\sphinxstyleemphasis{Modificar Usuarios}

\end{itemize}

\sphinxAtStartPar
Es decir, casi que tendremos un manejo básico de la Base de Datos, a través de una Interfaz.


\subsection{Base de Datos}
\label{\detokenize{documentos/instalacion:base-de-datos}}
\sphinxAtStartPar
Como Base de Datos, hemos empleado sqlite3. \sphinxhref{https://docs.python.org/es/3.10/library/sqlite3.html}{SQLite} , es una biblioteca de
C que provee una base de datos ligera basada en disco que no requiere un proceso de servidor separado y permite acceder
a la base de datos usando una variación no estándar del lenguaje de consulta SQL. Esta se convierte en una Base de Datos
perfecta para proyectos pequeños como este.

\sphinxAtStartPar
Para realizar una modificación en la Base de Datos, iremos al archivo \sphinxstylestrong{runsql.py}. Aqui encontraremos
este código:

\begin{sphinxVerbatim}[commandchars=\\\{\}]
\PYG{k+kn}{import} \PYG{n+nn}{sqlite3}

\PYG{k}{def} \PYG{n+nf}{databasecreate}\PYG{p}{(}\PYG{p}{)}\PYG{p}{:}

    \PYG{n+nb}{print}\PYG{p}{(}\PYG{l+s+s2}{\PYGZdq{}}\PYG{l+s+s2}{Creando la base de datos}\PYG{l+s+s2}{\PYGZdq{}}\PYG{p}{)}\PYG{p}{;}

    \PYG{c+c1}{\PYGZsh{} TABLA DE USUARIOS}

    \PYG{n}{conn} \PYG{o}{=} \PYG{n}{sqlite3}\PYG{o}{.}\PYG{n}{connect}\PYG{p}{(}\PYG{l+s+s1}{\PYGZsq{}}\PYG{l+s+s1}{bottle.db}\PYG{l+s+s1}{\PYGZsq{}}\PYG{p}{)}
    \PYG{n}{conn}\PYG{o}{.}\PYG{n}{execute}\PYG{p}{(}\PYG{l+s+s2}{\PYGZdq{}}\PYG{l+s+s2}{CREATE TABLE usuarios (nombre char(50) PRIMARY KEY NOT NULL, dni char(10) NOT NULL, pass char(20) NOT NULL, correo char(60) NOT NULL, departamento char(100), categoria char(5))}\PYG{l+s+s2}{\PYGZdq{}}\PYG{p}{)}
    \PYG{n}{conn}\PYG{o}{.}\PYG{n}{execute}\PYG{p}{(}\PYG{l+s+s2}{\PYGZdq{}}\PYG{l+s+s2}{INSERT INTO usuarios (nombre, dni, correo, departamento, categoria, pass) VALUES (}\PYG{l+s+s2}{\PYGZsq{}}\PYG{l+s+s2}{admin}\PYG{l+s+s2}{\PYGZsq{}}\PYG{l+s+s2}{,}\PYG{l+s+s2}{\PYGZsq{}}\PYG{l+s+s2}{admin}\PYG{l+s+s2}{\PYGZsq{}}\PYG{l+s+s2}{,}\PYG{l+s+s2}{\PYGZsq{}}\PYG{l+s+s2}{admin@uca.es}\PYG{l+s+s2}{\PYGZsq{}}\PYG{l+s+s2}{,}\PYG{l+s+s2}{\PYGZsq{}}\PYG{l+s+s2}{Administracion}\PYG{l+s+s2}{\PYGZsq{}}\PYG{l+s+s2}{,}\PYG{l+s+s2}{\PYGZsq{}}\PYG{l+s+s2}{ADM}\PYG{l+s+s2}{\PYGZsq{}}\PYG{l+s+s2}{, }\PYG{l+s+s2}{\PYGZsq{}}\PYG{l+s+s2}{admin}\PYG{l+s+s2}{\PYGZsq{}}\PYG{l+s+s2}{) }\PYG{l+s+s2}{\PYGZdq{}}\PYG{p}{)}

    \PYG{c+c1}{\PYGZsh{} TABLA DE ASIGNATURAS}

    \PYG{n}{conn}\PYG{o}{.}\PYG{n}{execute}\PYG{p}{(}\PYG{l+s+s2}{\PYGZdq{}}\PYG{l+s+s2}{CREATE TABLE asignatura (id INTEGER PRIMARY KEY, nombre char(100))}\PYG{l+s+s2}{\PYGZdq{}}\PYG{p}{)}

    \PYG{k}{while} \PYG{k+kc}{True}\PYG{p}{:}

        \PYG{n}{consulta} \PYG{o}{=} \PYG{n+nb}{input}\PYG{p}{(}\PYG{l+s+s2}{\PYGZdq{}}\PYG{l+s+s2}{Introduce una asignatura (n) para cancelar: }\PYG{l+s+s2}{\PYGZdq{}}\PYG{p}{)}

        \PYG{k}{if} \PYG{n}{consulta} \PYG{o}{==} \PYG{l+s+s1}{\PYGZsq{}}\PYG{l+s+s1}{n}\PYG{l+s+s1}{\PYGZsq{}}\PYG{p}{:}
            \PYG{k}{break}
        \PYG{k}{else}\PYG{p}{:}
            \PYG{n}{query} \PYG{o}{=} \PYG{l+s+s2}{\PYGZdq{}}\PYG{l+s+s2}{INSERT INTO asignatura (nombre) VALUES (}\PYG{l+s+s2}{\PYGZsq{}}\PYG{l+s+si}{\PYGZob{}\PYGZcb{}}\PYG{l+s+s2}{\PYGZsq{}}\PYG{l+s+s2}{)}\PYG{l+s+s2}{\PYGZdq{}}\PYG{o}{.}\PYG{n}{format}\PYG{p}{(}\PYG{n}{consulta}\PYG{p}{)}
            \PYG{n}{conn}\PYG{o}{.}\PYG{n}{execute}\PYG{p}{(}\PYG{n}{query}\PYG{p}{)}

    \PYG{c+c1}{\PYGZsh{} TABLA DE PROF\PYGZus{}ASIG}

    \PYG{n}{conn}\PYG{o}{.}\PYG{n}{execute}\PYG{p}{(}\PYG{l+s+s2}{\PYGZdq{}}\PYG{l+s+s2}{CREATE TABLE prof\PYGZus{}asig (profesor char(50), asignatura char(100), foreign key(profesor) references usuarios(nombre) );}\PYG{l+s+s2}{\PYGZdq{}}\PYG{p}{)}

    \PYG{n}{conn}\PYG{o}{.}\PYG{n}{commit}\PYG{p}{(}\PYG{p}{)}
    \PYG{n}{conn}\PYG{o}{.}\PYG{n}{close}\PYG{p}{(}\PYG{p}{)}
    \PYG{n+nb}{print}\PYG{p}{(}\PYG{l+s+s2}{\PYGZdq{}}\PYG{l+s+se}{\PYGZbs{}n}\PYG{l+s+s2}{ Base de datos creada}\PYG{l+s+s2}{\PYGZdq{}}\PYG{p}{)}


\PYG{k}{if} \PYG{n+nv+vm}{\PYGZus{}\PYGZus{}name\PYGZus{}\PYGZus{}} \PYG{o}{==} \PYG{l+s+s1}{\PYGZsq{}}\PYG{l+s+s1}{\PYGZus{}\PYGZus{}main\PYGZus{}\PYGZus{}}\PYG{l+s+s1}{\PYGZsq{}}\PYG{p}{:}

    \PYG{n}{databasecreate}\PYG{p}{(}\PYG{p}{)}
\end{sphinxVerbatim}

\begin{sphinxadmonition}{warning}{Advertencia:}
\sphinxAtStartPar
\sphinxstylestrong{No cambiar el esquema de las tablas, solo la información en ellas.}
\end{sphinxadmonition}

\sphinxAtStartPar
Para cambiar la información de las tablas, simplemente sustituiremos las columnas de correo y contraseña del admin, dejando
el nombre \sphinxstylestrong{admin} tal y como está. Así pues podremos tener un usuario admin, con nuestro correo y contraseña personalizado.


\chapter{Autores}
\label{\detokenize{documentos/autores:autores}}\label{\detokenize{documentos/autores::doc}}
\sphinxAtStartPar
\sphinxstylestrong{Álvaro Orellana Serrano}

\sphinxAtStartPar
\sphinxstylestrong{Zuleima Muñoz Jimenez}


\chapter{Ejecución}
\label{\detokenize{documentos/ejecucion:ejecucion}}\label{\detokenize{documentos/ejecucion::doc}}
\sphinxAtStartPar
Cuando abrimos el programa ante nosotros se nos muestra 3 posibles acciones:
\begin{enumerate}
\sphinxsetlistlabels{\arabic}{enumi}{enumii}{}{.}%
\item {} 
\sphinxAtStartPar
Puedes realizar una búsqueda de usuario.

\item {} 
\sphinxAtStartPar
Puedes registrarte.

\item {} 
\sphinxAtStartPar
Puedes iniciar sesión.

\end{enumerate}

\sphinxAtStartPar
A continuación, se muestra una descripción detallada paso a paso de cada una de ellas


\section{Realizar una búsqueda}
\label{\detokenize{documentos/ejecucion:realizar-una-busqueda}}
\sphinxAtStartPar
Si lo que se quiere es realizar una búsqueda, nos vamos al menú que
se muestra llamado \sphinxstyleemphasis{Búsqueda de Usuario}. En este nos dan las opciones de datos con los que podemos realizar una búsqueda:
\begin{itemize}
\item {} 
\sphinxAtStartPar
\sphinxstylestrong{Nombre:} Podemos realizar una búsqueda introduciendo el nombre de la persona que estamos buscando, ya sea el nombre, el apellido, o ambas cosas.

\item {} 
\sphinxAtStartPar
\sphinxstylestrong{Departamentos:} Se muestra una lista de los departamentos existentes y solo tenemos que seleccionar uno.

\item {} 
\sphinxAtStartPar
\sphinxstylestrong{Categorias:} Se muestra una lista de las categorías existentes y solo tenemos que seleccionar una.

\item {} 
\sphinxAtStartPar
\sphinxstylestrong{Asignaturas:} Se muestra una lista de las asignaturas existentes y solo tenemos que seleccionar una.

\item {} 
\sphinxAtStartPar
\sphinxstylestrong{DNI:} Podemos introducir el DNI completo de la persona que estamos buscando.

\end{itemize}

\sphinxAtStartPar
Podemos rellenar todas las opciones de búsqueda, solo algunas, o incluso ninguna. Una vez introducido los datos deseados,
presionamos el botón \sphinxstyleemphasis{Buscar} y se nos abrirá un apartado a la derecha de la pantalla con los resultados.


\section{Registrarse}
\label{\detokenize{documentos/ejecucion:registrarse}}
\sphinxAtStartPar
Si, por lo contrario, se quiere registrar. Pulsamos sobre la opción Regístrate y nos mostrará un menú preparado para ello.
El menú está compuesto por los siguientes apartados:
\begin{itemize}
\item {} 
\sphinxAtStartPar
\sphinxstylestrong{Nombre y Apellidos:} Se introduce el nombre y apellidos de la persona que se está registrando.

\item {} 
\sphinxAtStartPar
\sphinxstylestrong{Contraseña:} Nos pide una contraseña con la que posteriormente podremos identificarnos. Esta contraseña tiene que ser de un máximo de 20 caracteres.

\item {} 
\sphinxAtStartPar
\sphinxstylestrong{DNI:} Se introduce el DNI de la persona que se está registrando.

\item {} 
\sphinxAtStartPar
\sphinxstylestrong{Correo Electrónico:} Se requiere un correo electrónico perteneciente a la persona. Este posteriormente será nuestra forma de identificarnos cuando intentemos iniciar sesión.

\item {} 
\sphinxAtStartPar
\sphinxstylestrong{Seleccionar Departamento:} Se muestra una lista de los departamentos a los que puede pertenecer el usuario.

\item {} 
\sphinxAtStartPar
\sphinxstylestrong{Seleccionar Categoría:} Se muestra una lista de las categorías a los que puede pertenecer el usuario.
\begin{itemize}
\item {} 
\sphinxAtStartPar
Si en este apartado seleccionamos la categoría PDI, cuando pulsemos Registro se nos mostrará una lista de asignaturas a las que puede pertenecer el usuario. Puede seleccionar varias asignaturas.

\end{itemize}

\end{itemize}

\sphinxAtStartPar
Una vez hemos acabado de rellenar los datos, pulsamos Registro, se nos mostrará un mensaje de registro exitoso si los datos introducidos están bien,
y una vez acabado, pulsamos Inicio.


\section{Identificarse}
\label{\detokenize{documentos/ejecucion:identificarse}}
\sphinxAtStartPar
Si, finalmente, se quiere identificar. Pulsamos sobre la opción Identifícate y nos mostrará un menú preparado para ello. El menú está compuesto por los siguientes apartados:
\begin{itemize}
\item {} 
\sphinxAtStartPar
\sphinxstylestrong{Correo Electrónico:} Se requiere el correo electrónico que fue introducido cuando el usuario realizó el registro.

\item {} 
\sphinxAtStartPar
\sphinxstylestrong{Contraseña:} Se requiere la contraseña que fue introducida en el proceso de registro.

\end{itemize}

\sphinxAtStartPar
Al interactuar con este menú, puede pasar lo siguiente:
\begin{itemize}
\item {} 
\sphinxAtStartPar
Si metes un correo electrónico que no existe en la base de datos. Al intentar iniciar sesión te muestra un mensaje de error al introducir los datos, y vuelve a mostrarte el menú de inicio de sesión.

\item {} 
\sphinxAtStartPar
Si metes una contraseña incorrecta, muestra mensaje de error y vuelve a mostrar el menú de inicio de sesión.

\end{itemize}

\sphinxAtStartPar
Todo lo anteriormente explicado es desde el punto de vista de un usuario normal, pero existe un usuario denominado como administrador,
que al iniciar sesión tiene funciones especiales. El administrador además de realizar las búsquedas, puede realizar:
\begin{itemize}
\item {} 
\sphinxAtStartPar
\sphinxstylestrong{Modificar Usuario:} Con esta función el administrador puede modificar cualquier usuario (incluyendo a si mismo). Cuando accedes a esta opción se muestra lo siguiente:
\begin{itemize}
\item {} 
\sphinxAtStartPar
\sphinxstyleemphasis{Una lista con los usuarios existentes}, para seleccionar al usuario que se desea modificar.

\item {} 
\sphinxAtStartPar
\sphinxstyleemphasis{Contraseña}: Puede cambiar la contraseña de acceso del usuario.

\item {} 
\sphinxAtStartPar
\sphinxstyleemphasis{DNI}: Puede introducir un número nuevo.

\item {} 
\sphinxAtStartPar
\sphinxstyleemphasis{Correo Electrónico}: Puede cambiar el correo del usuario (hay que tener en cuenta que, si se modifica, el usuario debe usar el nuevo correo para acceder).

\item {} 
\sphinxAtStartPar
\sphinxstyleemphasis{Seleccionar Departamento}: Puede cambiar de departamento al usuario seleccionando uno de los que se muestra en la lista.

\item {} 
\sphinxAtStartPar
\sphinxstyleemphasis{Seleccionar Categoría}: Se muestra una lista con las categorías, y el administrador podrá seleccionar la que quiera asignarle al usuario.

\end{itemize}

\end{itemize}

\sphinxAtStartPar
Cuando todo este rellenado, se selecciona la opción Modificar y se modificaran los datos del usuario de forma exitosa.
\begin{itemize}
\item {} 
\sphinxAtStartPar
\sphinxstylestrong{Eliminar Usuario:} Con esta función el administrador puede eliminar cualquier usuario. Cuando selecciona esta opción, se le muestra la lista de usuarios, una vez seleccionado uno, se pulsa Eliminar, y se eliminará el usuario de forma exitosa.

\end{itemize}


\chapter{Documentación del Código}
\label{\detokenize{documentos/codigo:documentacion-del-codigo}}\label{\detokenize{documentos/codigo::doc}}

\section{Runsql}
\label{\detokenize{documentos/codigo:runsql}}

\bigskip\hrule\bigskip



\section{Run}
\label{\detokenize{documentos/codigo:run}}


\renewcommand{\indexname}{Índice}
\printindex
\end{document}